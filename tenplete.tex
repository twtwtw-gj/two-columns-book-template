%%%%プリアンプルはじめ%%%%%%%%%%%%%%%%%%%%%%%%%%%%%%%%%%%%%%%%%%%%%%%%%%%%%%%%%%%%%%%%%%%%%%%%%%%%%%%%%%%%%%%%%%%%%%%%%%%%%%%%%%%%%

\documentclass[twocolumn,titlepage]{tarticle}

%二段組縦書き、タイトルページ出力を指定している。

\topmargin-2.3cm\advance\textwidth3.0cm
\oddsidemargin-.5cm\advance\textheight-.5cm

\usepackage{pxrubrica} %ルビ振り
\usepackage{url} %URL
\usepackage{amsmath} %複雑な数学の表記が可能になる。複雑な物でないのなら、別にこのパッケージをプリアンブルに入れなくてもできる。
\usepackage{amsfonts}
\setcounter{tocdepth}{1} %目次のレベルをどこまで鮮明にするか指定(subsectionも表すことができる)。
\usepackage{comment} %コメント(後述)の執筆ができる


\title{} %かっこ内にタイトルを書く。英語でも日本語でもよし。
\author{} %かっこ内に著者名を書く。
\date{} %かっこ内に日付を書く。日付入力の必要がないときは手を付けずコンパイルする。もしこれをなくしてコンパイルすると、生成ファイルにコンパイルした日の日付がプリントされてしまう。


%%%%プリアンプルおわり%%%%%%%%%%%%%%%%%%%%%%%%%%%%%%%%%%%%%%%%%%%%%%%%%%%%%%%%%%%%%%%%%%%%%%%%%%%%%%%%%%%%%%%%%%%%%%%%%%%%%%%%%%%%


\begin{document} %ドキュメント開始。

\maketitle %タイトルを出力する。

\clearpage 
%改ページの命令。まだ出力していない図表を総て出力する。
%二段組だと上下片一方がなくなってしまうが、newpageでも可能。

\tableofcontents
%これがあると後の文章をコンピューターが読み込んで勝手に目次を作り、出力してくれる。
%コンピューターはコンパイルするときに文章を読み込むため、テキストを(手を加えずに)2回以上コンパイルしないと、ちゃんとファイルに反映されない。

\clearpage

\part{} %部。かっこの中に文章を書くと部のタイトルとして出力される。

\section{} %小節。

\section*{タイトル} 
%は出したいが、小節としてナンバリングしたくない、目次に出力したくないときこうする。

\subsection{} %小々節

%%%%本文開始%%%%%%%%%%%%%%%%%%%%%%%%%%%%%%%%%%%%%%%%%%%%%%%%%%%%%%%%%%%%%%%%%%%%%%%%%%%%%

ここに本文を書く。

一行開けて改行すると、\LaTeX に「改行する」と認識される。

ルビはこのように\jruby{振}{ふ}る。途中に[g]を挟み込むことで、\jruby[g]{大量にルビを振る}{いっぱいルビ}(親字やルビの文字数に関係なくバランスよくルビを振る)
ことも\jruby[g]{可能}{キャンドゥ}になる。

\textbf{太字}にすることもできるし、\bou{傍点}をつけることもできる。

\noindent ←空白を詰めるにはこうする。

% コマンド\input{}や\include{}を使うと、ファイルを分割することができる。

\begin{comment}
コメントの記述である。
これを挟むと間に何を書いても、生成ファイルに反映されなくなる。
これを挟まなくても、行頭に%をつけてコメント扱いにすることもできる。
執筆や処理の上で気になった部分をメモするのに使える。
\end{comment}

%%%%本文おわり%%%%%%%%%%%%%%%%%%%%%%%%%%%%%%%%%%%%%%%%%%%%%%%%%%%%%%%%%%%%%%%%%%%%%%%%%%%%

\end{document} %ドキュメント終了。

ここでは何を書いてもpdfファイルに反映されない。